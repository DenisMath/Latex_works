\documentclass[a4paper,12pt]{article}
\usepackage[T2A]{fontenc}
\usepackage[utf8]{inputenc} % любая желаемая кодировка
\usepackage[russian,english]{babel}
%\usepackage[pdftex,unicode]{hyperref}
\usepackage{indentfirst} % включить отступ у первого абзаца
\usepackage[all]{xy}
\usepackage{amssymb}
\usepackage{amsmath}
\usepackage{amsfonts}

\title{Абстрактная схема игрового процесса}
\author{BetLab}
\date{2014}

\begin{document} % начало документа

\maketitle % заголовок

%\section*{Class 1}{Вычислимое и невычислимое}

% Игра включает в себя время игры. Изменение состояния игры.

% На множестве игр введем множество описаний игр. 

% Описание игры - шаблон. Расширение множества шаблонов. Анкета - столбец базы данных.

\section*{Вступление}
Формализация абстрактного понятия игры мотивировано созданием в полученной формалистике описательного языка, предоставляющего возможность четкой постановки задач, исключающей возможность неоднозначной трактовки. 

Минимальные требования : описания должны быть конечными текстами; способ составления описания игры должен быть вычислимой функцией, т.е. задаваться алгоритмом. 

\begin{displaymath}
\xymatrix{
    *+[F]{\text{Игра}}  \ar@{-->}[d]  \ar[r] & *+[F]{\text{Описание}}  \ar@{-->}[d]         \\
    *+[F]{\text{Текст}}\ar[r] &  *+[F]{\text{Смысл}}
}
\end{displaymath}


\section{Базовые определения}
Будем рассматривать множество игр $\mathcal{G}$ и множество слов (текстов\footnote{Текст является словом в алфавите, дополненном символами форматирования - пробел, переход строки и т.д.}) $T_{\omega}$ над алфавитом\footnote{Алфавитом можно считать набор символов, при помощи которых записываются программы.} \label{footnote1} $\omega$.


{\it Схемой описания } назовем тройку $(\mathcal{G}, T_{\omega}, \alpha : \mathcal{G} \rightarrow T_{\omega} )$( дальше просто {\it схема }). 

На множестве схем введем отношение эквивалентности. Назовем схемы $(\mathcal{G}, T_{\omega}, \alpha) $ и  $(\mathcal{G}, T_{\omega}, \beta) $ эквивалентными ( $(\mathcal{G}, T_{\omega}, \alpha) \simeq (\mathcal{G}, T_{\omega}, \beta) $), если существует биекция $f:   T_{\omega} \rightarrow T_{\omega} $, такая, что 
$ \alpha \circ f = \beta $.

Также введем на множестве схем отношение порядка $\succcurlyeq$. Будем говорить, что 
  $$(\mathcal{G}, T_{\omega}, \alpha)\succcurlyeq (\mathcal{G}, T_{\omega}, \beta) $$
 если существует функция $f:   T_{\omega} \rightarrow T_{\omega} $, такая, что следующая диаграмма коммутативна:


\begin{displaymath}
    \xymatrix{
                            &      &  T_{\omega} \ar[dd]^{f} \\
        {\mathcal{G}} \ar[rru]^{\alpha} \ar[rrd]_{\beta}&      &               \\
                            &       & T_{\omega}  }
\end{displaymath}

Такую функцию $f$ будем называть трансляцией схемы $(\mathcal{G}, T_{\omega}, \alpha) $ в схему $(\mathcal{G}, T_{\omega}, \beta) $. Множество трансляций является полугруппой относительно операции суперпозиции.

Заметим, что 
$$(\mathcal{G}, T_{\omega}, \alpha) \simeq (\mathcal{G}, T_{\omega}, \beta) \iff ((\mathcal{G}, T_{\omega}, \alpha)\succcurlyeq (\mathcal{G}, T_{\omega}, \beta))\cup ((\mathcal{G}, T_{\omega}, \alpha)\preccurlyeq (\mathcal{G}, T_{\omega}, \beta))   $$ 

 Будем говорить, что схема $(\mathcal{G}, T_{\omega}, \alpha)$ уточняет схему $(\mathcal{G}, T_{\omega}, \beta)$, если
 имеет место:
$$(\mathcal{G}, T_{\omega}, \alpha)\succcurlyeq (\mathcal{G}, T_{\omega}, \beta). $$

Дальше будем рассматривать схемы с точностью до эквивалентности. На множестве классов эквивалентности естественным
образом вводится отношение $\succ$:

 если представители классов эквивалентности находятся в отношении $\succcurlyeq$ и при этом не принадлежат одному классу эквивалентности, то соответствующие классы находятся в отношении $\succ$. 

Описанные классы эквивалентности будем называть {\it обобщенными схемами}.

Таким образом имеем отношение строгого порядка на частично упорядоченном множестве обобщенных схем.

\section{Полное описание}

Назовем схему описания $(\mathcal{G},T_{\omega},\alpha)$ {\it полной}, если $\alpha$ - инъекция. Соответствующую данной
схеме обобщенную схему также будем называть полной. Дальше будем исходить из допущения, что для данных $\langle\mathcal{G},T_{\omega}\rangle$ полная схема существует.

Назовем схему описания $(\mathcal{G},T_{\omega},\alpha)$ {\it пустой}, если $E_{\alpha} = \varnothing $. 

\begin{theorem}
Множество обобщенных схем, содержащее полную и пустую схемы является полной решеткой.
\end{theorem}


% $$\begin{xy}
%   \newdir{ >}{{}*!/-2ex/\dir{>}}
% \xymatrix{
%   {\kappa} \ar@{ >->}[d]_{!} \ar@{ >->}[r]^{!} & {1} \ar[d]_{T_{\mu}} \ar[r]^{id} & {1} \ar[d]^{T} \\
%   {1} \ar[r]^{T_{\alpha}} & {\Omega} \ar[r]^{\neg_{\mu}} & {\Omega}  \\
%   {\alpha} \ar@{ >->}[u]^{!} \ar@{ >->}[r]^{!} & {1} \ar[u]_{T}}
% \end{xy}$$

\end{document} % конец документа